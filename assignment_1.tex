\documentclass{article}

\usepackage[scientific-notation=true, binary-units=true]{siunitx}
\sisetup{per-mode=fraction}%
\sisetup{scientific-notation=false}%

\title{SYSC 4502 Assignment 1}
\date{February 5th, 2017}
\author{Jessica Morris \(100882290\)}

\begin{document}

\maketitle

\begin{enumerate}

\item
\begin{enumerate}

\item Replacing a single IP processing queue with queues-per-socket helps by eliminating the need for a software interrupt after the hardware interrupt. The responsibility for sending the packet to a destination queue is shifted to the hardware interrupt handler, which reduces the amount of context-switching the CPU must perform. One of the issues with implementing this, is that it will add overhead to the hardware interrupt handler, as it will now have to do the IP and TCP processing determine which of the destination buffers will receive a packet.

\item Implementing protocol processing as part of the application, instead of having a software interrupt, would also reduce the amount of context-switching the CPU has to do. However, this method is ineffective for two reasons. Firstly, this would increase the size and overhead of all socket-using applications, as they would now be responsible for IP and TCP processing. Secondly, this would slow down applications, as they would have to wait until they have priority over the CPU to do the packet processing.

\end{enumerate}

\item The thread pool architecture works best for short requests. Since creating a new thread requires more overhead than assigning an idle thread from a pool, thread pools work well when there is a large number of small tasks to run. The new-thread-per-packet architecture is better only when there are few, long-running tasks to run; in which case, spawning a large pool of threads to re-use would not have any inherent benefit over spawning threads as needed.

\item
\begin{enumerate}

\item Persistent connections are the default behaviour of HTTP connections. The "Connection" header field can be used by the client or the server to terminate a persistent connection.

\item HTTP by itself does not provide encryption.

\item Yes, a client can open three or more simultaneous connections with a given server. (Note that RFC 2616 recommends that they shouldn't have more than two connections at a time.)

\item Because HTTP is stateless, the client doesn't know the server's state and the server doesn't know the client's state. Therefore, it is possible that one side closes the connection for being idle for too long, while the other side just finishes processing and starts to send data.

\end{enumerate}

\item
\begin{enumerate}

\item http://gaia.cs.umass.edu/cs453/index.html

\item HTTP 1.1

\item A persistent connection (the last header line is Connection:keep-alive).

\item The IP address of the host is not included in a HTTP request, so this is indeterminate.

\item Mozilla Netscape v7.2

\item The browser needs to be known in order to determine the object's compatibility.

\end{enumerate}

\item Let $ T_p $ denote the propagation delay of the channel.
$$ T_p = \frac{\SI{10}{\metre}}{\SI{3e8}{\metre\per\second}} = \SI{0.03}{\micro\second} $$

Firstly, consider the time taken to download the objects over parallel, non-persistent connections. The total time taken to transmit an object over a non-persistent connection is given by the time taken to send a request to connect, plus the time to send an acknowledgement, plus the time taken to request an object, plus the time taken to send the object. Let $ T_1(s, b) $ represent the download time for an object over this connection, where $ s $ is the size of the object being downloaded, and $ b $ is the bitrate for the download:
$$ T_1(s,b) = (T_{connect} + T_p) + (T_{acknowledge} + T_p) + (T_{request} + T_p) + (T_{object} + T_p) $$

Which in this case, simplifies to:
$$ T_1(s,b) = (\frac{\SI{200}{\bit}}{\SI[parse-numbers = false]{b}{\bit\per\second}} + T_p) + (\frac{\SI{200}{\bit}}{\SI[parse-numbers = false]{b}{\bit\per\second}} + T_p) + (\frac{\SI{200}{\bit}}{\SI[parse-numbers = false]{b}{\bit\per\second}} + T_p) + (\frac{\SI[parse-numbers = false]{s}{\bit}}{\SI[parse-numbers = false]{b}{\bit\per\second}} + T_p) $$
$$ T_1(s,b) = \frac{600 + s}{b} + 4T_p $$

Thus, the total time for the request is given by the time taken for downloading the first object, and the time for downloading one of the 10 referenced objects:
$$ T_{total} = T_1(\SI{100}{\kilo\bit}, \SI{150}{\bit\per\second}) + T_1(\SI{100}{\kilo\bit}, \SI{15}{\bit\per\second}) $$
$$ T_{total} = \SI{7377}{\second} + 8T_p $$

For a persistent HTTP connection, the setup, acknowledge, and request packets are not needed for the 10 referenced objects. So, the time for a persistent connection, with downloads in series, is given by:
$$ T_{total} = T_1(\SI{100}{\kilo\bit}, \SI{150}{\bit\per\second}) + 10(\frac{\SI{200}{\bit}}{\SI{150}{\bit\per\second}} + T_p + \frac{\SI{100}{\kilo\bit}}{\SI{150}{\bit\per\second}} + T_p) $$
$$ T_{total} = \SI{7351}{\second} + 24T_p $$

Since $ T_p $ is negligible compared to the transmission delay, and $ \Delta T = 0.35\% $, the persistent HTTP connection does not have a significant advantage over doing parallel downloads over non-persistent connection.

\item
\begin{enumerate}

\item A WHOIS is a public database which allows you to find information regarding domains, such as who it is registered to, who the registrar is, and who you can contact about the domain name. WHOIS databases are run by registrars and registries; for example, the Public Interest Registry maintains the .org registry.

\item Using whois.cira.ca: one of google.ca's Domain Name Servers is ns1.google.com (216.239.32.10). One of duckduckgo.com's Domain Name Servers is ns0.dnsmadeeasy.com (208.94.148.2).

\item The local Carleton DNS server is phoenix.sce.carleton.ca, address 134.117.56.1.

The A-type queries simply returned the address record for the server. For example, querying ns0.dnsmadeeasy.com returned the address 208.94.148.2.

The MX- and NS-type queries returned much more information than A-type queries. They both returned information such as responsible mail server (for the SCE network, this is naren.sce.carleton.ca), the refresh rate, and the default TTL.

The difference between MX- and NS-type queries is that NS-type will only return information on the mail server, whereas MX-type queries return information on all DNS servers for a domain.

\item The Carleton web server only has one IP address. I was unable to locate a web server with multiple IP addresses.

\item 134.117.0.0 - 134.117.255.255 (n.b. Carleton does not appear on the ARIN whois database, so I used nirsoft.net to find this.)

\item The Carleton University DNS servers are: ns1.carleton.ca, ns2.carleton.ca, ns1.d-zone.ca and ns2.d-zone.ca

\item An attacker can use WHOIS databases and the nslookup tool to determine the IP address of a web site's server. Once they have obtained the IP address, they can flood the web server with requests in what is called a Denial-of-Service attack. A server under attack will be unable to service other clients' requests, because it has been bogged down by requests from the attacker. A reasonable web server should limit the number of requests from one client over a specific time period, but that would not stop a Distributed Denial-of-Service attack.

\item WHOIS databases should be publicly available. They are the primary reference for finding the contact information of the registrants of domain names, as well as the IP blocks registered to those domains. This is useful if there is an issue such as a domain using more IPs than its designated block, and a primary point of contact must be located.

\end{enumerate}

\end{enumerate}
\end{document}
